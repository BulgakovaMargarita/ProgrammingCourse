\documentclass[12pt,a4paper]{report}
\usepackage[utf8]{inputenc}
\usepackage[russian]{babel}
\usepackage[OT1]{fontenc}
\usepackage{amsmath}
\usepackage{amsfonts}
\usepackage{amssymb}
\usepackage{graphicx}
\usepackage{cmap}					% поиск в PDF
\usepackage{mathtext} 				% русские буквы в формулах
%\usepackage{tikz-uml}               % uml диаграммы

% TODOs
\usepackage[%
  colorinlistoftodos,
  shadow
]{todonotes}

% Генератор текста
\usepackage{blindtext}

%------------------------------------------------------------------------------

% Подсветка синтаксиса
\usepackage{color}
\usepackage{xcolor}
\usepackage{listings}
 
 % Цвета для кода
\definecolor{string}{HTML}{B40000} % цвет строк в коде
\definecolor{comment}{HTML}{008000} % цвет комментариев в коде
\definecolor{keyword}{HTML}{1A00FF} % цвет ключевых слов в коде
\definecolor{morecomment}{HTML}{8000FF} % цвет include и других элементов в коде
\definecolor{captiontext}{HTML}{FFFFFF} % цвет текста заголовка в коде
\definecolor{captionbk}{HTML}{999999} % цвет фона заголовка в коде
\definecolor{bk}{HTML}{FFFFFF} % цвет фона в коде
\definecolor{frame}{HTML}{999999} % цвет рамки в коде
\definecolor{brackets}{HTML}{B40000} % цвет скобок в коде
 
 % Настройки отображения кода
\lstset{
language=C, % Язык кода по умолчанию
morekeywords={*,...}, % если хотите добавить ключевые слова, то добавляйте
 % Цвета
keywordstyle=\color{keyword}\ttfamily\bfseries,
stringstyle=\color{string}\ttfamily,
commentstyle=\color{comment}\ttfamily\itshape,
morecomment=[l][\color{morecomment}]{\#}, 
 % Настройки отображения     
breaklines=true, % Перенос длинных строк
basicstyle=\ttfamily\footnotesize, % Шрифт для отображения кода
backgroundcolor=\color{bk}, % Цвет фона кода
%frame=lrb,xleftmargin=\fboxsep,xrightmargin=-\fboxsep, % Рамка, подогнанная к заголовку
frame=tblr
rulecolor=\color{frame}, % Цвет рамки
tabsize=3, % Размер табуляции в пробелах
showstringspaces=false,
 % Настройка отображения номеров строк. Если не нужно, то удалите весь блок
numbers=left, % Слева отображаются номера строк
stepnumber=1, % Каждую строку нумеровать
numbersep=5pt, % Отступ от кода 
numberstyle=\small\color{black}, % Стиль написания номеров строк
 % Для отображения русского языка
extendedchars=true,
literate={Ö}{{\"O}}1
  {Ä}{{\"A}}1
  {Ü}{{\"U}}1
  {ß}{{\ss}}1
  {ü}{{\"u}}1
  {ä}{{\"a}}1
  {ö}{{\"o}}1
  {~}{{\textasciitilde}}1
  {а}{{\selectfont\char224}}1
  {б}{{\selectfont\char225}}1
  {в}{{\selectfont\char226}}1
  {г}{{\selectfont\char227}}1
  {д}{{\selectfont\char228}}1
  {е}{{\selectfont\char229}}1
  {ё}{{\"e}}1
  {ж}{{\selectfont\char230}}1
  {з}{{\selectfont\char231}}1
  {и}{{\selectfont\char232}}1
  {й}{{\selectfont\char233}}1
  {к}{{\selectfont\char234}}1
  {л}{{\selectfont\char235}}1
  {м}{{\selectfont\char236}}1
  {н}{{\selectfont\char237}}1
  {о}{{\selectfont\char238}}1
  {п}{{\selectfont\char239}}1
  {р}{{\selectfont\char240}}1
  {с}{{\selectfont\char241}}1
  {т}{{\selectfont\char242}}1
  {у}{{\selectfont\char243}}1
  {ф}{{\selectfont\char244}}1
  {х}{{\selectfont\char245}}1
  {ц}{{\selectfont\char246}}1
  {ч}{{\selectfont\char247}}1
  {ш}{{\selectfont\char248}}1
  {щ}{{\selectfont\char249}}1
  {ъ}{{\selectfont\char250}}1
  {ы}{{\selectfont\char251}}1
  {ь}{{\selectfont\char252}}1
  {э}{{\selectfont\char253}}1
  {ю}{{\selectfont\char254}}1
  {я}{{\selectfont\char255}}1
  {А}{{\selectfont\char192}}1
  {Б}{{\selectfont\char193}}1
  {В}{{\selectfont\char194}}1
  {Г}{{\selectfont\char195}}1
  {Д}{{\selectfont\char196}}1
  {Е}{{\selectfont\char197}}1
  {Ё}{{\"E}}1
  {Ж}{{\selectfont\char198}}1
  {З}{{\selectfont\char199}}1
  {И}{{\selectfont\char200}}1
  {Й}{{\selectfont\char201}}1
  {К}{{\selectfont\char202}}1
  {Л}{{\selectfont\char203}}1
  {М}{{\selectfont\char204}}1
  {Н}{{\selectfont\char205}}1
  {О}{{\selectfont\char206}}1
  {П}{{\selectfont\char207}}1
  {Р}{{\selectfont\char208}}1
  {С}{{\selectfont\char209}}1
  {Т}{{\selectfont\char210}}1
  {У}{{\selectfont\char211}}1
  {Ф}{{\selectfont\char212}}1
  {Х}{{\selectfont\char213}}1
  {Ц}{{\selectfont\char214}}1
  {Ч}{{\selectfont\char215}}1
  {Ш}{{\selectfont\char216}}1
  {Щ}{{\selectfont\char217}}1
  {Ъ}{{\selectfont\char218}}1
  {Ы}{{\selectfont\char219}}1
  {Ь}{{\selectfont\char220}}1
  {Э}{{\selectfont\char221}}1
  {Ю}{{\selectfont\char222}}1
  {Я}{{\selectfont\char223}}1
  {і}{{\selectfont\char105}}1
  {ї}{{\selectfont\char168}}1
  {є}{{\selectfont\char185}}1
  {ґ}{{\selectfont\char160}}1
  {І}{{\selectfont\char73}}1
  {Ї}{{\selectfont\char136}}1
  {Є}{{\selectfont\char153}}1
  {Ґ}{{\selectfont\char128}}1
  {\{}{{{\color{brackets}\{}}}1 % Цвет скобок {
  {\}}{{{\color{brackets}\}}}}1 % Цвет скобок }
}
 
 % Для настройки заголовка кода
\usepackage{caption}
\DeclareCaptionFont{white}{\color{сaptiontext}}
\DeclareCaptionFormat{listing}{\parbox{\linewidth}{\colorbox{сaptionbk}{\parbox{\linewidth}{#1#2#3}}\vskip-4pt}}
\captionsetup[lstlisting]{format=listing,labelfont=white,textfont=white}
\renewcommand{\lstlistingname}{Код} % Переименование Listings в нужное именование структуры


%------------------------------------------------------------------------------

\author{М.В.Булгакова}
\title{Программирование}
\begin{document}
\listoftodos
\maketitle
\chapter{Основные конструкции языка}
%############################################################
\section{Задание 1}
\subsection{Задание}
Пользователь задает три корня кубического уравнения 
\begin{equation}
x^3+bx^2+cx+d 
\end{equation}
 (например, 1, 2, 3). Вывести значения b, c и d, например: b=-6, c=11, d=-6.
\subsection{Теоритические сведения}

С помощью \verb+main.c+, находящейся в многомодульном проекте \verb+subproject +, можно задать параметр запуска для автоматического выполнения\\ 
\verb+coefficients_of_equations.c+, находящейся в статической библиотеке \verb+lib+, в параметрах нужно указать 
\verb+--is-coefficients_of_equation+ и через пробелы 3 значения, равные корням кубического уравнения. Так же при задании значения параметра запуска в виде\verb+--interactive+ включается интерактивный режим, где данная функция принимает значения, вводимые пользователем программы, выбор выполнения данной задачи описан в \verb+main_menu.c+, где использовались операторы условного перехода switch.Ввод и вывод данных в пользовательском режиме происходет в подпроекте \verb+app+ в 
\verb+coefficients_of_equation.c+, заголовочным файлом которого является \verb+coefficients_of_equation.h+. \\ 
Были созданы модульные тесты в подпроекте \verb+test+. 

\subsection{Проектирование}
В main.c у пользователя запрашивают режим работы программы, состоящий из:
\begin{enumerate}
\item \verb+--interactive+ - Ручной ввод значений
\item \verb+--is-coefficients_of_equation+- Автоматическая работа, через ввод параметров запуска
\end{enumerate}

В \verb+coefficients_of_equation.c+, находящейся в подпроекте \verb+app+,  реализовано взаимодейтсвие с пользователем, считывая введенные значения с консоли, и передавая их в \verb+coefficients_of_equations.c+, находящейся в статической библиотеке \verb+lib+, где производится поиск коэффициентов уравнений.\\
Во время автоматической работы в \verb+coefficients_of_equations.c+ значения передаются из параметров запуска.\\
Модульные тесты находятся в \verb+test+ \verb+tst_testtest.cpp+.\\
Листинги \verb+main.c+ и \verb+main_menu.c+ приведены в приложении
\subsection{Описание тестового стенда и методики тестирования}
Среда разработки QtCreator 3.5.0, компилятор GCC 4.8.4 (x86 64 bit), операционная система Linux Mint 17.2 Cinnamon 64 bit.
В процессе выполнения задания производилось ручное тестирование.
Модульное тестирование реализовано при помощи фреймворка QtTest.

\subsection{Тестовый план и результаты тестирования}
При вызове автоматического теста программа обращается к к методу класса TestTest\\ \verb+ test_i_coefficient_of_equation_function()+,\\
\verb+ test_j_coefficient_of_equation_function()+,\\
\verb+ test_k_coefficient_of_equation_function()+, в которой по уже заданным значениям производится поиск коэффициентов и сравнение рузельтатов с помощью процедуры\verb+  QCOMPARE+.
Для модульного теста, в котором 
\begin{equation}
x_1=x_2=x_3=2,
\end{equation}, все три коэффициента равны 10. Тест прошел успешно.
Для ручного теста, в котором \begin{equation}x_1=3; x_2=2; x_3=1, i=-6; j=11; k=-6.\end{equation} Тест прошел успешно.
Листинги модульных тестов приведены в приложении.

\subsection{Выводы}

При написании данной работы были приобретены навыки работы с отладкой(debug), навыки создания модульных тестов и умение разбивать задачи на подзадачи, отделяя общение с пользователем от бизнес-логики, и создание многомодульных проектов. 

\subsection*{Листинги}

\lstinputlisting[]
{../sources/subproject/app/coefficients_of_equation.c}

\lstinputlisting[]
{../sources/subproject/app/coefficients_of_equation.h}

\lstinputlisting[]
{../sources/subproject/lib/coefficients_of_equations.c}

\lstinputlisting[]
{../sources/subproject/lib/search_coefficients_of_equation_function.h}


%############################################################

\section{Задание 2}
\subsection{Задание}
Нa шахматной доске стоят черный король и три белые ладьи (ладья бьет по горизонтали и вертикали). Определить, не находится ли король под боем, а если есть угроза, то от кого именно. Координаты короля и ладей вводить целыми числами.
\subsection{Теоритические сведения}
С помощью \verb+main.c+, находящейся в многомодульном проекте \verb+subproject +, можно задать параметр запуска для автоматического выполнения\\ 
\verb+treat_to_king_of_chesss.c+, находящейся в статической библиотеке \verb+lib+, в параметрах нужно указать 
\verb+--is-poisk_ugrozi+ и через пробелы 8 значения, равные координатам короля и ладей. Так же при задании значения параметра запуска в виде\verb+--interactive+ включается интерактивный режим, где данная функция принимает значения, вводимые пользователем программы, выбор выполнения данной задачи описан в \verb+main_menu.c+, где использовались операторы условного перехода switch.Ввод и вывод данных в пользовательском режиме происходет в подпроекте \verb+app+ в 
\verb+treat_to_king_of_chess.c+, заголовочным файлом которого является \verb+treat_to_king_of_chess.h+. \\ 
Были созданы модульные тесты в подпроекте \verb+test+. 

\subsection{Проектирование}
В main.c у пользователя запрашивают режим работы программы, состоящий из:
\begin{enumerate}
\item \verb+--interactive+ - Ручной ввод значений
\item \verb+--is-poisk_ugrozi+- Автоматическая работа, через ввод параметров запуска
\end{enumerate}

В \verb+treat_to_king_of_chess.c+, находящейся в подпроекте \verb+app+,  реализовано взаимодейтсвие с пользователем, считывая введенные значения с консоли, и передавая их в \verb+treat_to_king_of_chesss.c+, находящейся в статической библиотеке \verb+lib+, где производится поиск коэффициентов уравнений.\\
Во время автоматической работы в \verb+treat_to_king_of_chesss.c+ значения передаются из параметров запуска.\\
Модульные тесты находятся в \verb+test+ \verb+tst_testtest.cpp+.\\
Листинги \verb+main.c+ и \verb+main_menu.c+ приведены в приложении

\subsection{Описание тестового стенда и методики тестирования}
Среда разработки QtCreator 3.5.0, компилятор GCC 4.8.4 (x86 64 bit), операционная система Linux Mint 17.2 Cinnamon 64 bit.
В процессе выполнения задания производилось ручное тестирование.
Модульное тестирование реализовано при помощи фреймворка QtTest.

\subsection{Тестовый план и результаты тестирования}
При вызове автоматического теста программа обращается к методу класса TestTest \\ \verb+ void test_treat_to_king_of_chess_function()+, в которой по уже заданным значениям производится поиск коэффициентов и сравнение рузельтатов с помощью процедуры\verb+  QCOMPARE+.
Для модульного теста, в котором координаты короля 
\begin{equation}
x=1,y=3
\end{equation},
координаты первой ладьи
\begin{equation}
x=2,y=2
\end{equation} ,
координаты второй ладьи
\begin{equation}
x=2,y=1
\end{equation},
координаты третьей ладьи
\begin{equation}
x=4,y=5
\end{equation}
угроза королю от второй ладьи. Тест прошел успешно.
Для ручного теста, в котором \begin{equation}king_x=1, king_y=1, rook1_x=1, rook1_y=5, rook2_x=10, rook2_y=10, rook3_x=3, rook3_y=15.\end{equation}, угроза исходит от первой ладьи .Тест прошел успешно.
Листинги модульных тестов приведены в приложении.

\subsection{Выводы}
При написании данной работы были улучшены навыки работы с отладкой(debug), навыки создания модульных тестов и умение разбивать задачи на подзадачи, отделяя общение с пользователем от бизнес-логики.
\subsection*{Листинги}

\lstinputlisting[]
{../sources/subproject/app/treat_to_king_of_chess.c}

\lstinputlisting[]
{../sources/subproject/app/treat_to_king_of_chess.h}

\lstinputlisting[]
{../sources/subproject/lib/treat_to_king_of_chesss.c}

\lstinputlisting[]
{../sources/subproject/lib/treat_to_king_of_chess_function.h}

%############################################################
\chapter{Циклы}
\section{Задание 1}
\subsection{Задание}
Составить из соответствующих цифр чисел M и N наибольшее возможное число.
Примеры: 4157, 8024 > 8157; 323, 10714 > 10724.
\subsection{Теоритические сведения}
С помощью \verb+main.c+, находящейся в многомодульном проекте \verb+subproject +, можно задать параметр запуска для автоматического выполнения\\ 
\verb+max_composite_numbers.c+, находящейся в статической библиотеке \verb+lib+, в параметрах нужно указать 
\verb+--is-max_vozmojnoe+ и через пробелы 2 значения, равные значениям двух чисел M и N. В \verb+max_composite_numbers.c+ используется цикл while, математические операции pow, floor, fmod. Так же при задании значения параметра запуска в виде\verb+--interactive+ включается интерактивный режим, где данная функция принимает значения, вводимые пользователем программы, выбор выполнения данной задачи описан в \verb+main_menu.c+, где использовались операторы условного перехода switch.Ввод и вывод данных в пользовательском режиме происходет в подпроекте \verb+app+ в 
\verb+max_composite_number.c+, заголовочным файлом которого является \verb+max_composite_number.h+. \\ 
Были созданы модульные тесты в подпроекте \verb+test+. 
\subsection{Проектирование}
В main.c у пользователя запрашивают режим работы программы, состоящий из:
\begin{enumerate}
\item \verb+--interactive+ - Ручной ввод значений
\item \verb+--is-max_vozmojnoe+- Автоматическая работа, через ввод параметров запуска
\end{enumerate}

В \verb+max_composite_number.c+, находящейся в подпроекте \verb+app+,  реализовано взаимодейтсвие с пользователем, считывая введенные значения с консоли, и передавая их в \verb+max_composite_numbers.c+, находящейся в статической библиотеке \verb+lib+, где производится поиск коэффициентов уравнений.\\
Во время автоматической работы в \verb+max_composite_numbers.c+ значения передаются из параметров запуска.\\
Модульные тесты находятся в \verb+test+ \verb+tst_testtest.cpp+.\\
Листинги \verb+main.c+ и \verb+main_menu.c+ приведены в приложении
\subsection{Описание тестового стенда и методики тестирования}
Среда разработки QtCreator 3.5.0, компилятор GCC 4.8.4 (x86 64 bit), операционная система Linux Mint 17.2 Cinnamon 64 bit.
В процессе выполнения задания производилось ручное тестирование.
Модульное тестирование реализовано при помощи фреймворка QtTest.

\subsection{Тестовый план и результаты тестирования}
При вызове автоматического теста программа обращается к методу класса TestTest \\ \verb+ void test_max_composite_number_function()+, в которой по уже заданным значениям производится поиск коэффициентов и сравнение рузельтатов с помощью процедуры\verb+  QCOMPARE+.
Для модульного теста, в котором  
\begin{equation}
M=1038,N=5147
\end{equation},
Максимальное возможное составное число - 5148. Тест прошел успешно.
Для ручного теста, в котором \begin{equation}M=38,N=500
\end{equation}, максимальное возможное составное число - 538 .Тест прошел успешно.
Листинги модульных тестов приведены в приложении.
\subsection{Выводы}
При написании данной работы были получены навыки работы со стандартной библиотекой math.h, навыки создания циклов while.
\subsection*{Листинги}
\lstinputlisting[]
{../sources/subproject/app/max_composite_number.c}

\lstinputlisting[]
{../sources/subproject/app/max_composite_number.h}

\lstinputlisting[]
{../sources/subproject/lib/max_composite_numbers.c}

\lstinputlisting[]
{../sources/subproject/lib/finding_max_composite_number_function.h}

%############################################################

\chapter{Массивы}
\section{Задание 1}
\subsection{Задание}
Каждый элемент вектора A(n) (кроме двух крайних) заменить выражением: \begin{equation}
a_i = (a_(i-1) + 2a_i + a_(i+1)) / 4
\end{equation}, а крайние элементы – выражениями: \begin{equation}a_1 = (a_1 + a_2)/2, a_n = (a_(n-1) + a_n)/2 \end{equation}.
\subsection{Теоритические сведения}
С помощью \verb+main.c+, находящейся в многомодульном проекте \verb+subproject +, можно задать параметр запуска для автоматического выполнения\\ 
\verb+replacement_of_elements_in_array.c+, находящейся в подпроекте \verb+app+, в параметрах нужно указать 
\verb+--is-zamena_elementov_mass+. Данная программа работает с файлами, таким образом с клавиатуры необходимо будет ввести путь к файлу, в котором хранится элементы массива. В \verb+replacement_of_elements_in_array.c+ используется цикл for, функции работы с памятью free, malloc и функции работы с файлами: fopen, fscanf, fclose. Так же при задании значения параметра запуска в виде\verb+--interactive+ включается интерактивный режим, где данная функция принимает значения равное пути к файлу, вводимое пользователем программы, выбор выполнения данной задачи описан в \verb+main_menu.c+, где использовались операторы условного перехода switch.

\subsection{Проектирование}

В \verb+replacement_of_elements_in_array.c+  реализовано взаимодейтсвие с пользовательским файлом, считывая введенные значения с файла, указанного пользователем, \verb+replacement_of_elements_in_array.c+ производит замену элементов по заданному условию.\\

Листинги \verb+main.c+ и \verb+main_menu.c+ приведены в приложении
\subsection{Описание тестового стенда и методики тестирования}
Среда разработки QtCreator 3.5.0, компилятор GCC 4.8.4 (x86 64 bit), операционная система Linux Mint 17.2 Cinnamon 64 bit.
В процессе выполнения задания производилось ручное тестирование.
\subsection{Тестовый план и результаты тестирования}
Во время выполнения ручных тестов сбоев не происходило, программа меняла значения элементов массива в соответствии с условием.
\subsection{Выводы}
При написании данной работы были получены навыки работы с файлами и массивами.
\subsection*{Листинги}

\lstinputlisting[]
{../sources/subproject/app/replacement_of_elements_in_array.c}

\lstinputlisting[]
{../sources/subproject/app/replacement_of_elements_in_array.h}
%############################################################
\chapter{Строки}
\section{Задание 1}
\subsection{Задание}
Текст, не содержащий собственных имен и сокращений, набран полностью прописными русскими буквами. Заменить все прописные буквы, кроме букв, стоящих после точки, строчными буквами.
\subsection{Теоритические сведения}
С помощью \verb+main.c+, находящейся в многомодульном проекте \verb+subproject +, можно задать параметр запуска для автоматического выполнения\\ 
\verb+sentence_to_lower.c+, находящейся в подпроекте \verb+app+, в параметрах нужно указать 
\verb+--is-sentence_to_lower+. Данная программа работает с файлами, таким образом с клавиатуры необходимо будет ввести путь к файлу, в котором хранится элементы массива, и путь к файлу, в который будет записываться отредактированный текст. В \verb+sentence_to_lower.c+ используется цикл for, функция работы с символами tolower и функции работы с файлами: fopen, fgetc,  fputc, fclose. Так же при задании значения параметра запуска в виде\verb+--interactive+ включается интерактивный режим, где данная функция принимает значения равное пути к двум файлам, вводимые пользователем программы, выбор выполнения данной задачи описан в \verb+main_menu.c+, где использовались операторы условного перехода switch.

\subsection{Проектирование}

В \verb+sentence_to_lower.c+  реализовано взаимодейтсвие с пользовательским файлом, считывая введенные значения с файла, указанного пользователем, \verb+sentence_to_lower.c+ производит замену прописных букв на строчные и записывает во второй файл, указанный пользователем.\\

Листинги \verb+main.c+ и \verb+main_menu.c+ приведены в приложении
\subsection{Описание тестового стенда и методики тестирования}
Среда разработки QtCreator 3.5.0, компилятор GCC 4.8.4 (x86 64 bit), операционная система Linux Mint 17.2 Cinnamon 64 bit.
В процессе выполнения задания производилось ручное тестирование.
\subsection{Тестовый план и результаты тестирования}
Во время выполнения ручных тестов сбоев не происходило, программа меняла предложения в соответствии с условием.
\subsection{Выводы}
При написании данной работы были получены навыки работы с файлами и символами.
\subsection*{Листинги}

\lstinputlisting[]
{../sources/subproject/app/sentence_to_lower.c}

\lstinputlisting[]
{../sources/subproject/app/sentence_to_lower.h}
%############################################################
\chapter{Стек}
\section{Задание 1}
\subsection{Задание}
Для класса стек реализовать конструктор, конструктор копирования, деструктор, pop, push методы.
\subsection{Теоритические сведения}
Класс Stack находится в подпроекте \verb+stack+. В заголовочном файле \verb+stack.h+ описан класс, а в исполняемом \verb+stack.cpp+ файле реализованы методы класса и конструкторы с деструктором. В \verb+main.cpp+ реализовано общение с пользователем и вызов методов и полей класса.

\subsection{Проектирование}

В \verb+main.cpp+  реализовано взаимодейтсвие с пользовательским файлом, считывая введенные значения с файла, указанного пользователем, метод класса Stack \verb+pop+ записывает элементы в поле класса \verb+array+, метод \verb+push+ "собирает" число, состоящее из элементов массива с последнего элемента до нулевого.\\
Пример. \begin{equation}
а_1=2; а_2=3; а_3=4;
\end{equation} 
число - \begin{equation}431 \end{equation} 

\subsection{Описание тестового стенда и методики тестирования}
Среда разработки QtCreator 3.5.0, компилятор GCC 4.8.4 (x86 64 bit), операционная система Linux Mint 17.2 Cinnamon 64 bit.
В процессе выполнения задания производилось ручное тестирование.
\subsection{Тестовый план и результаты тестирования}
Во время выполнения ручных тестов произошли сбои в процессе присваивания в методе класса \verb+pop+.
\subsection{Выводы}
При написании данной работы были получены навыки работы с классами.
\subsection*{Листинги}

\lstinputlisting[]
{../sources/subproject/stack/main.cpp}

\lstinputlisting[]
{../sources/subproject/stack/stack.h}
\lstinputlisting[]
{../sources/subproject/stack/stack.cpp}
%############################################################
\chapter{Приложение}
\subsection*{Листинги}

\lstinputlisting[]
{../sources/subproject/app/main.c}

\lstinputlisting[]
{../sources/subproject/app/main_menu.c}

\lstinputlisting[]
{../sources/subproject/app/main_menu.h}

\lstinputlisting[]
{../sources/subproject/test/tst_testtest.cpp}

\end{document}