\documentclass[12pt,a4paper]{report}
\usepackage[utf8]{inputenc}
\usepackage[russian]{babel}
\usepackage[OT1]{fontenc}
\usepackage{amsmath}
\usepackage{amsfonts}
\usepackage{amssymb}
\usepackage{graphicx}
\usepackage{cmap}					% поиск в PDF
\usepackage{mathtext} 				% русские буквы в формулах
%\usepackage{tikz-uml}               % uml диаграммы

% TODOs
\usepackage[%
  colorinlistoftodos,
  shadow
]{todonotes}

% Генератор текста
\usepackage{blindtext}

%------------------------------------------------------------------------------

% Подсветка синтаксиса
\usepackage{color}
\usepackage{xcolor}
\usepackage{listings}
 
 % Цвета для кода
\definecolor{string}{HTML}{B40000} % цвет строк в коде
\definecolor{comment}{HTML}{008000} % цвет комментариев в коде
\definecolor{keyword}{HTML}{1A00FF} % цвет ключевых слов в коде
\definecolor{morecomment}{HTML}{8000FF} % цвет include и других элементов в коде
\definecolor{captiontext}{HTML}{FFFFFF} % цвет текста заголовка в коде
\definecolor{captionbk}{HTML}{999999} % цвет фона заголовка в коде
\definecolor{bk}{HTML}{FFFFFF} % цвет фона в коде
\definecolor{frame}{HTML}{999999} % цвет рамки в коде
\definecolor{brackets}{HTML}{B40000} % цвет скобок в коде
 
 % Настройки отображения кода
\lstset{
language=C, % Язык кода по умолчанию
morekeywords={*,...}, % если хотите добавить ключевые слова, то добавляйте
 % Цвета
keywordstyle=\color{keyword}\ttfamily\bfseries,
stringstyle=\color{string}\ttfamily,
commentstyle=\color{comment}\ttfamily\itshape,
morecomment=[l][\color{morecomment}]{\#}, 
 % Настройки отображения     
breaklines=true, % Перенос длинных строк
basicstyle=\ttfamily\footnotesize, % Шрифт для отображения кода
backgroundcolor=\color{bk}, % Цвет фона кода
%frame=lrb,xleftmargin=\fboxsep,xrightmargin=-\fboxsep, % Рамка, подогнанная к заголовку
frame=tblr
rulecolor=\color{frame}, % Цвет рамки
tabsize=3, % Размер табуляции в пробелах
showstringspaces=false,
 % Настройка отображения номеров строк. Если не нужно, то удалите весь блок
numbers=left, % Слева отображаются номера строк
stepnumber=1, % Каждую строку нумеровать
numbersep=5pt, % Отступ от кода 
numberstyle=\small\color{black}, % Стиль написания номеров строк
 % Для отображения русского языка
extendedchars=true,
literate={Ö}{{\"O}}1
  {Ä}{{\"A}}1
  {Ü}{{\"U}}1
  {ß}{{\ss}}1
  {ü}{{\"u}}1
  {ä}{{\"a}}1
  {ö}{{\"o}}1
  {~}{{\textasciitilde}}1
  {а}{{\selectfont\char224}}1
  {б}{{\selectfont\char225}}1
  {в}{{\selectfont\char226}}1
  {г}{{\selectfont\char227}}1
  {д}{{\selectfont\char228}}1
  {е}{{\selectfont\char229}}1
  {ё}{{\"e}}1
  {ж}{{\selectfont\char230}}1
  {з}{{\selectfont\char231}}1
  {и}{{\selectfont\char232}}1
  {й}{{\selectfont\char233}}1
  {к}{{\selectfont\char234}}1
  {л}{{\selectfont\char235}}1
  {м}{{\selectfont\char236}}1
  {н}{{\selectfont\char237}}1
  {о}{{\selectfont\char238}}1
  {п}{{\selectfont\char239}}1
  {р}{{\selectfont\char240}}1
  {с}{{\selectfont\char241}}1
  {т}{{\selectfont\char242}}1
  {у}{{\selectfont\char243}}1
  {ф}{{\selectfont\char244}}1
  {х}{{\selectfont\char245}}1
  {ц}{{\selectfont\char246}}1
  {ч}{{\selectfont\char247}}1
  {ш}{{\selectfont\char248}}1
  {щ}{{\selectfont\char249}}1
  {ъ}{{\selectfont\char250}}1
  {ы}{{\selectfont\char251}}1
  {ь}{{\selectfont\char252}}1
  {э}{{\selectfont\char253}}1
  {ю}{{\selectfont\char254}}1
  {я}{{\selectfont\char255}}1
  {А}{{\selectfont\char192}}1
  {Б}{{\selectfont\char193}}1
  {В}{{\selectfont\char194}}1
  {Г}{{\selectfont\char195}}1
  {Д}{{\selectfont\char196}}1
  {Е}{{\selectfont\char197}}1
  {Ё}{{\"E}}1
  {Ж}{{\selectfont\char198}}1
  {З}{{\selectfont\char199}}1
  {И}{{\selectfont\char200}}1
  {Й}{{\selectfont\char201}}1
  {К}{{\selectfont\char202}}1
  {Л}{{\selectfont\char203}}1
  {М}{{\selectfont\char204}}1
  {Н}{{\selectfont\char205}}1
  {О}{{\selectfont\char206}}1
  {П}{{\selectfont\char207}}1
  {Р}{{\selectfont\char208}}1
  {С}{{\selectfont\char209}}1
  {Т}{{\selectfont\char210}}1
  {У}{{\selectfont\char211}}1
  {Ф}{{\selectfont\char212}}1
  {Х}{{\selectfont\char213}}1
  {Ц}{{\selectfont\char214}}1
  {Ч}{{\selectfont\char215}}1
  {Ш}{{\selectfont\char216}}1
  {Щ}{{\selectfont\char217}}1
  {Ъ}{{\selectfont\char218}}1
  {Ы}{{\selectfont\char219}}1
  {Ь}{{\selectfont\char220}}1
  {Э}{{\selectfont\char221}}1
  {Ю}{{\selectfont\char222}}1
  {Я}{{\selectfont\char223}}1
  {і}{{\selectfont\char105}}1
  {ї}{{\selectfont\char168}}1
  {є}{{\selectfont\char185}}1
  {ґ}{{\selectfont\char160}}1
  {І}{{\selectfont\char73}}1
  {Ї}{{\selectfont\char136}}1
  {Є}{{\selectfont\char153}}1
  {Ґ}{{\selectfont\char128}}1
  {\{}{{{\color{brackets}\{}}}1 % Цвет скобок {
  {\}}{{{\color{brackets}\}}}}1 % Цвет скобок }
}
 
 % Для настройки заголовка кода
\usepackage{caption}
\DeclareCaptionFont{white}{\color{сaptiontext}}
\DeclareCaptionFormat{listing}{\parbox{\linewidth}{\colorbox{сaptionbk}{\parbox{\linewidth}{#1#2#3}}\vskip-4pt}}
\captionsetup[lstlisting]{format=listing,labelfont=white,textfont=white}
\renewcommand{\lstlistingname}{Код} % Переименование Listings в нужное именование структуры


%------------------------------------------------------------------------------

\author{М.В.Булгакова}
\title{Программирование}
\begin{document}
\listoftodos
\maketitle
\chapter{Основные конструкции языка}
%############################################################
\section{Задание 1}
\subsection{Задание}
Пользователь задает три корня кубического уравнения 
%\begin{enumerate}
$x^3+bx^2+cx+d $
%\end{enumerate}
 (например, 1, 2, 3). Вывести значения b, c и d, например: b=-6, c=11, d=-6.
\subsection{Теоритические сведения}

При выполнинии задания в main.c использовался switch для предаставления выбора пользователю вида теста программы(ручной ввод или автоматический). В \begin{verbatim}poisk_znacheniy.c \end{verbatim} использовались операторы условного перехода if и циклы for для нахождения b, c и d.

\subsection{Проектирование}
В main.c у пользователя запрашивают режим работы программы, состоящий из:
\begin{enumerate}
\item Ручной ввод значений
\item Автоматические тесты
\end{enumerate}

В \begin{verbatim}poisk_znacheniy.c \end{verbatim}  реализовано взаимодейтсвие с пользователем, считывая введенные значения с консоли, \begin{verbatim}poisk_znacheniy.c \end{verbatim}  производит поиск коэффициентов уравнений.
Модульные тесты находятся в test.c.
\subsection{Описание тестового стенда и методики тестирования}

Для создания проекта использовались Qt Creator 3.5.0 (opensource) и GCC.
Пользователь может выбрать один из режимов работы программы

\begin{enumerate}
\item Ручной ввод значений
\item Автоматические тесты
\end{enumerate}

Автоматические тесты, на подобие модульных, контролируют исправность программы.

\subsection{Тестовый план и результаты тестирования}
При вызове автоматического теста программа обращается к процедуре\begin{verbatim} void automate_test_variant7_1()\end{verbatim}. Данная процедура вызывает процедуру \begin{verbatim}void test_poisk_variant7_1()\end{verbatim}, в которой по уже заданным значениям производится поиск коэффициентов и сравнение рузельтатов с помощью процедуры\begin{verbatim} void test_result_variant7_1(int expected, int actual)\end{verbatim}, которая выводит на экран "Ok", если полученное значение совпало с ожидаемым, в противном случае выводит "Test fail".

\subsection{Выводы}

При написании данной работы были приобретены навыки работы с отладкой(debug), навыки создания модульных тестов и умение разбивать задачи на подзадачи, отделяя общение с пользователем от бизнес-логики.

\subsection*{Листинги}
\lstinputlisting[]
{../sources/helloworld/main.c}

\lstinputlisting[]
{../sources/helloworld/poisk_znacheniy.c}

\lstinputlisting[]
{../sources/helloworld/poisk_znacheniy.h}

\lstinputlisting[]
{../sources/helloworld/test.c}

\lstinputlisting[]
{../sources/helloworld/test.h}


%############################################################

\section{Задание 2}
\subsection{Задание}
Нa шахматной доске стоят черный король и три белые ладьи (ладья бьет по горизонтали и вертикали). Определить, не находится ли король под боем, а если есть угроза, то от кого именно. Координаты короля и ладей вводить целыми числами.
\subsection{Теоритические сведения}
При выполнинии задания в main.c использовался switch для предаставления выбора пользователю вида теста программы(ручной ввод или автоматический). В \begin{verbatim}poisk_ugrozi.c \end{verbatim} использовались операторы условного перехода if для нахождения угрозы королю от ладьи.
\subsection{Проектирование}
В main.c у пользователя запрашивают режим работы программы, состоящий из:
\begin{enumerate}
\item Ручной ввод значений
\item Автоматические тесты
\end{enumerate}

В \begin{verbatim}poisk_ugrozi.c \end{verbatim}  реализовано взаимодейтсвие с пользователем, считывая введенные значения с консоли, \begin{verbatim}poisk_ugrozi.c \end{verbatim}  производит поиск угрозы королю от одной или нескольких ладей.
Модульные тесты находятся в test.c.
\subsection{Описание тестового стенда и методики тестирования}
Для создания проекта использовались Qt Creator 3.5.0 (opensource) и GCC.
Пользователь может выбрать один из режимов работы программы

\begin{enumerate}
\item Ручной ввод значений
\item Автоматические тесты
\end{enumerate}

Автоматические тесты, на подобие модульных, контролируют исправность программы.
\subsection{Тестовый план и результаты тестирования}
При вызове автоматического теста программа обращается к процедуре\begin{verbatim} void automate_test_variant7_2()\end{verbatim}. Данная процедура вызывает процедуру \begin{verbatim}void test_poisk_variant7_2()\end{verbatim}, в которой по уже заданным значениям производится поиск номера ладьи, угрожающей королю, и сравнение рузельтатов с помощью процедуры\begin{verbatim} void test_result_variant7_2(int expected, int actual)\end{verbatim}, которая выводит на экран "Ok", если полученное значение совпало с ожидаемым, в противном случае выводит "Test fail".
\subsection{Выводы}
При написании данной работы были улучшены навыки работы с отладкой(debug), навыки создания модульных тестов и умение разбивать задачи на подзадачи, отделяя общение с пользователем от бизнес-логики.
\subsection*{Листинги}
\lstinputlisting[]
{../sources/helloworld/main.c}

\lstinputlisting[]
{../sources/helloworld/poisk_ugrozi.c}

\lstinputlisting[]
{../sources/helloworld/poisk_ugrozi.h}

\lstinputlisting[]
{../sources/helloworld/test.c}

\lstinputlisting[]
{../sources/helloworld/test.h}

%############################################################
\chapter{Циклы}
\section{Задание 1}
\subsection{Задание}
Составить из соответствующих цифр чисел M и N наибольшее возможное число.
Примеры: 4157, 8024 > 8157; 323, 10714 > 10724.
\subsection{Теоритические сведения}
При выполнинии задания в main.c использовался switch для предаставления выбора пользователю вида теста программы(ручной ввод или автоматический). В \begin{verbatim}max_vozmojnoe.c \end{verbatim} использовались операторы условного перехода if и switch, циклы while и математические функции floor, fmod, pow  для нахождения наибольшего возможного числа.
\subsection{Проектирование}
В main.c у пользователя запрашивают режим работы программы, состоящий из:
\begin{enumerate}
\item Ручной ввод значений
\item Автоматические тесты
\end{enumerate}

В \begin{verbatim}max_vozmojnoe.c \end{verbatim}  реализовано взаимодейтсвие с пользователем, считывая введенные значения с консоли, \begin{verbatim}max_vozmojnoe.c \end{verbatim}  производит поиск ниабольшего возможного числа, составленного из чисел M и N.
Модульные тесты находятся в test.c.
\subsection{Описание тестового стенда и методики тестирования}
Для создания проекта использовались Qt Creator 3.5.0 (opensource) и GCC.
Пользователь может выбрать один из режимов работы программы

\begin{enumerate}
\item Ручной ввод значений
\item Автоматические тесты
\end{enumerate}

Автоматические тесты, на подобие модульных, контролируют исправность программы.
\subsection{Тестовый план и результаты тестирования}
При вызове автоматического теста программа обращается к процедуре\begin{verbatim} void automate_test_max_vozmojnoe()\end{verbatim}. Данная процедура вызывает процедуру \begin{verbatim}void test_poisk_max_vozmojnoe()\end{verbatim}, в которой по уже заданным значениям производится поиск наибольшего возможного числа, состоящего из M и N, и сравнение рузельтатов с помощью процедуры\begin{verbatim} void test_result_max_vozmojnoe(int expected, int actual)\end{verbatim}, которая выводит на экран "Ok", если полученное значение совпало с ожидаемым, в противном случае выводит "Test fail".
\subsection{Выводы}
При написании данной работы были получены навыки работы со стандартной библиотекой math.h, навыки создания циклов while.
\subsection*{Листинги}
\lstinputlisting[]
{../sources/helloworld/main.c}

\lstinputlisting[]
{../sources/helloworld/max_vozmojnoe.c}

\lstinputlisting[]
{../sources/helloworld/max_vozmojnoe.h}

\lstinputlisting[]
{../sources/helloworld/test.c}

\lstinputlisting[]
{../sources/helloworld/test.h}

%############################################################

\chapter{Массивы}
\section{Задание 1}
\subsection{Задание}
Каждый элемент вектора A(n) (кроме двух крайних) заменить выражением: $a_i = (a_(i-1) + 2a_i + a_(i+1)) / 4$, а крайние элементы – выражениями: $a_1 = (a_1 + a_2)/2, a_n = (a_(n-1) + a_n)/2$.
\subsection{Теоритические сведения}
При выполнинии задания в main.c использовался switch для предаставления выбора пользователю вида теста программы(ручной ввод или автоматический). В \begin{verbatim}zamena_elementov_mass.c \end{verbatim} использовались циклы for и функции работы с файлами fscanf, fopen, fclose и процедуры задания и освобождения памяти malloc и free.
\subsection{Проектирование}
В main.c у пользователя запрашивают режим работы программы, состоящий из:
\begin{enumerate}
\item Ручной ввод значений
\item Автоматические тесты
\end{enumerate}

В \begin{verbatim}zamena_elementov_mass.c \end{verbatim}  реализовано взаимодейтсвие с пользовательским файлом, считывая введенные значения с файла zamena.txt, \begin{verbatim}zamena_elementov_mass.c \end{verbatim}  производит замену элементов по заданному условию.
Модульные тесты находятся в test.c.
\subsection{Описание тестового стенда и методики тестирования}
Для создания проекта использовались Qt Creator 3.5.0 (opensource) и GCC.
Пользователь может выбрать один из режимов работы программы

\begin{enumerate}
\item Ручной ввод значений
\item Автоматические тесты
\end{enumerate}

Автоматические тесты, на подобие модульных, контролируют исправность программы.
\subsection{Тестовый план и результаты тестирования}
При вызове автоматического теста программа обращается к процедуре\begin{verbatim} void automate_test_zamena_elementov_mass()\end{verbatim}. Данная процедура вызывает процедуру \begin{verbatim}void test_poisk_zamena_elementov_mass()\end{verbatim}, в которой по уже заданным в файле значениям производит замену значений элементов массива, и сравнение рузельтатов с помощью процедуры\begin{verbatim} void test_result_zamena_elementov_mass()(int expected, int actual)\end{verbatim}, которая выводит на экран "Ok", если полученное значение совпало с ожидаемым, в противном случае выводит "Test fail".
\subsection{Выводы}
При написании данной работы были получены навыки работы с файлами и массивами.
\subsection*{Листинги}
\lstinputlisting[]
{../sources/helloworld/main.c}

\lstinputlisting[]
{../sources/helloworld/zamena_elementov_mass.c}

\lstinputlisting[]
{../sources/helloworld/zamena_elemetov_mass.h}

\lstinputlisting[]
{../sources/helloworld/test.c}

\lstinputlisting[]
{../sources/helloworld/test.h}

%############################################################


\end{document}